\section{Conclusiones}
\label{chap:conclusiones}

Este proyecto nos ha permitido comprobar la importancia y las ventajas de los lenguajes orientados a objetos, que permiten una programación más 
ágil y eficiente que los lenguajes estructurados. Esto permite expresar aspectos de la vida cotidiana de forma más realista.\\

Como \lluvia{} es un framework multihilo gestionado por señales, permite añadir a Javascript funciones nuevas, fundamentales para el desarrollo 
del proyecto. Esto permite enviar y recibir mensajes entre distintos componentes de la aplicación, mejorando la respuesta a los diferentes 
eventos. Además, la utilización de caracteres personales autónomos (Boids), nos ha permitido comprobar que se puede modelar artificialmente el 
comportamiento de un animal a partir de operaciones sencillas con vectores.\\

Modificando la aceleración en función de un estímulo se pueden generar comportamientos de huida, en los que se aumenta la aceleración hasta 
su máximo durante un espacio para ir reduciéndose de manera gradual a medida que se aleja del estímulo.
Si, además, se le añade un radio de visión, compuesto de radio y ángulo, también se pueden generar otro tipo de comportamientos.\\

Según el ángulo de visión, se generan comportamientos, los cuales pueden variar en función de los hábitos alimenticios de 
una determinada especie de Boid, por ejemplo. Es decir, si un Boid es carnívoro, tendrán los ojos ubicados en la parte delantera de la cabeza,
mientras que si es herbívoro, éstos estarán situados en los lados. Esta variación del ángulo de visión permite ajustar lo observado en la vida 
real a la simulación artificial.\\

El radio de visión permite determinar la distancia que ve el Boid. Éste, por ejemplo, puede variar en función de la edad, siendo mayor cuando
más joven es el Boids. También permite determinar si una especie tiene comportamientos de manada o no, en función de cuánto se acerquen o 
alejen de otros animales similares dentro de su radio de visión. De la misma manera, un Boid solo con un comportamiento de manada programado 
tenderá a acercarse a otros miembros cercanos de la manada.\\

Tanto para la visión como para los otros atributos del Boid, se ha comprobado que cambios de valor demasiado grandes en las variables 
generan comportamientos que no son naturales, demasiado ordenados en comparación con aquellos que existen en la naturaleza. Son las variaciones
pequeñas entre valores de un atributo las que devuelven comportamientos reales. Por ejemplo, variando levemente el valor del radio en el 
comportamiento de separación y el valor de la aceleración en el comportamiento de alienación se obtiene, de un comportamiento de bandada 
de pájaros, otro de grupo de peces. \\

Se ha observado también que los Boid oveja responden alejándose en grupo y en línea recta cuando el Boid cerdito se acerca a ellos
por la parte trasera moviéndose en zig-zag. Y si el Boid cerdito se sitúa en el punto de equilibrio (cuartos delanteros) del Boid oveja, 
los Boid oveja retroceden.\\

Los resultados obtenidos en este proyecto pueden proporcionar una herramienta para la simulación de comportamientos. Lo que en este
proyecto se basa en animales puede ser extendido a personas o partículas, generando una plataforma de experimentación.


