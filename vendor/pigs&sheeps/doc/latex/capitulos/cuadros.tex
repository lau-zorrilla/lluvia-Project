\section{Cuadros}
\label{sec:cuadros}

\begin{table}[h]
\caption{Cuadro de Comportamientos.}
\centering
\begin{tabular}{| p{4cm} | p{7cm} | p{4cm} |} % centered columns (4 columns)

\hline\hline %inserts double horizontal lines
Nombre de la técnica & Boid cerdito & Boid oveja \\ [0.5ex] % inserts table
%heading
\hline % inserts single horizontal line
% inserting body of the table
Técnica del limpiaparabrisas & El pastor debe moverse en zig-zag detrás de la manada, para mantenerlos en línea recta. & Las ovejas deben de agruparse y moverse en línea recta.\\

Zona de fuga & El cerdito se encuentra en la zona de fuga de la oveja. & La oveja se agita y se enfrenta a él. \\

Moverlos en una manga & El cerdo debe de situarse enfrente del punto de equilibrio & Las ovejas avanzan hacia atrás. \\

Sacar a las ovejas del corral con un controlador & El cerdo se sitúa  a 90º detrás del ganado. Los movimientos deben de ser perpendiculares a los del ganado, hacia delante y 
hacia atrás sobre la barra transversal de una gigante T. & las ovejas salen en manada del corral. \\ [1ex] % [1ex] adds vertical space
\hline %inserts single line
\end{tabular}
\label{table:nonlin} % is used to refer this table in the text
\end{table}