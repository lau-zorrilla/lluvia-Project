\section{Gates y Devices}
\label{sec:gates_devices}

\subsection{Introducción}
\label{subsection:intro_gates}

El servicio de mensajería de \lluvia{} está gestionado por las clases Gate y Device.

Diagrama de Gates y Device. Ver figura \ref{fig:figura511} (página \pageref{fig:figura511})

\subsubsection{Clase Device}
\label{subsubsection:device}

La clase Device provee un mecanismo asíncrono de comunicación con otros Devices. No tiene conexión propia con el DOM de HTML, pero ésta se 
realiza mediante un objeto de la clase Gate. Los Devices reciben eventos, que guardan en una cola de mensajes. En el atributo self\_ events es 
donde se almacena la lista de mensajes que éste puede enviar. Por ejemplo, el Device menuHandler atenderá a los siguientes eventos: 
get\_panel\_out(), restart\_game(), keep\_menu\_out() y get\_menu\_in().\\

\begin{verbatim}
this.self_events = [ "get_panel_out", "restart_game", 
                     "keep_menu_out", "get_menu_in" ].
\end{verbatim}

Cuando se realice  el envío de alguno de esos mensajes será necesario la utilización del método fireEvent y newMessage indicando el tipo de
mensaje, el nombre del mensaje y el evento al que se refiere a través de un gate. El siguiente ejemplo muestra el envío de un mensaje del tipo 
síncrono “restart\_game”: 

\begin{verbatim}
that.newGate("restart_option", Gate, { do_onclick: 
     function(event, element) {
            this.device.fireEvent(this.device.newMessage("sync",
            "restart_game", this))
        } })
\end{verbatim}

Además, para atender a cada uno de esos mensajes habrá que crear una función 'attend':\

\begin{verbatim}
MenuHandler.prototype.attend_keep_menu_out = function(date, msg) {
//Respuesta al mensaje recibido keep_menu_out de la cola de mensajes.
this.menu_effects.menu_automata.currentState.requested = 
                              this[this.view].menu_automata.state.out
}
\end{verbatim}

Cada vez que se crea un nuevo Gate, se añade un puerto al Device para que sea posible la comunicación.

\begin{verbatim}
this.handler.addPort("restart_game", this.world)
    countdown(this.world);
\end{verbatim}


A su vez los devices pueden enviar mensajes a otros devices lanzando un nuevo evento de envío de mensaje con los fireEvent y newMessage.
Por ejemplo:

\begin{verbatim}
this.device.fireEvent(this.device.newMessage("sync", 
                                  "restart_game", this))
\end{verbatim}


\subsubsection{Clase Gate}
\label{subsubsection:gate}
 
La clase Gate se encargan de gestionar eventos. Algunos son onClick(), onMouseOver() o onMouseOut(), que responden a la pulsación del ratón, a 
su paso por encima de un elemento y a su salida de él, respectivamente. 
Un objeto Gate está contenido dentro de un objeto Device y en su creación recibe como parámetros opcionales los siguiente elementos:

\begin{itemize}
 \item Identificador del elemento HTML que va a responder al evento (por ejemplo una etiqueta div) 
 \item Contenedor HTML donde situar el objeto Gate
 \item Acciones de respuesta.
\end{itemize}

Por ejemplo, si en el documento HTML tenemos el siguiente elemento:
\begin{verbatim}
     <div id='button_bar'>&nbsp;</div>
\end{verbatim}
Escribiríamos:
\begin{verbatim}
   that.newGate("instructions_option", Gate, {do_onclick: 
       function(event, element) {
            alert("Move the little pig to place sheeps 
                  into the barnyard")
   }
\end{verbatim}

Lo que genera que ante un evento, en este caso on\_click, sobre el div con identificador instructions\_option, se produzca  una respuesta como 
puede ser un mensaje emergente (alert).

Otra manera posible manera de crear un Gate sería la siguiente, en la que crearíamos una subclase (en el ejemplo, la clase Animation) de la 
clase Gate.En su creación recibe como parámetro “element” el elemento HTML del cual ha de recibir los eventos y en su función initialize hemos 
de llamar al supercontructor de Gate, con el método call() para heredar sus métodos y atributos.
Posteriormente, crearemos las funciones que se deben ejecutar al producirse los eventos. Ver código \ref{lst:code5211} 
(página \pageref{lst:code5211}), \ref{lst:code5122} (página \pageref{lst:code5122}) y \ref{lst:code5122} 
(página \pageref{lst:code5122})


Por último, dentro del Device que va a contener el Gate, llamaremos al método newGate para crear el objeto de la subclase que hemos creado, 
indicando el elemento HTML sobre el que actúa y el nombre de nuestra subclase:

\begin{verbatim}
that.menu_effects = that.newGate("menu", Animation)
\end{verbatim}


\subsection{Menú y Submenú}
\label{subsection:menu}

\subsubsection{Clase MenuAutomata}
\label{subsubsection:menu_automata}

Controla los estados de entrada y salida del menú. Deriva de la clase threadAutomata y tiene una serie de estados posibles:\\
'out', 'getting\_out', 'getting\_in', 'inside'.\\
Estos estados, a su vez, pueden encontrarse en estado anterior (previous), actual (current) o solicitado (requested). Según cual sea  el estado 
actual (current), se realizarán la funciones correspondientes.\\

Los estados irán cambiando de currentState desde ”requested” a “current” y posteriormente a “previous”.

\begin{itemize}
 \item Estado out: Cambia la altura de menú a 250px, que es la longitud del menú una vez está totalmente desplegado, incluyendo el submenú correspondiente a los 
 niveles. Ver código \ref{lst:code5211} (página \pageref{lst:code5211})

 \item Estado getting\_out: Incrementa la altura de menú de 5 en 5px  mientras este mida menos de 206 px y más de 50px, para que se despliegue poco hasta salir del todo.
 Ver código \ref{lst:code5212} (página \pageref{lst:code5212})

 \item Estado getting\_in: Decrementa la altura de menú de 5 en 5px, mientras éste mida  más de 55px para que de forma gradual vuelva a ocultarse. 
 La altura mínima es corresponde a la altura del botón menu. Ver código \ref{lst:code5213} (página \pageref{lst:code5213})
\end{itemize}


\subsubsection{Clase Animation}
\label{subsubsection:animation}

Recoge los eventos que se producen en el menú a partir de la interacción del usuario. 
Animation deriva de la clase Gate.
Recibe el elemento HTML que envuelve.
En su función initialize() además de crear como atributo propio el elemento (HTML) y de llamar al super constructor de Gate para 
heredar sus metodos y atributos, asocia los efectos de menuAutomata al elemento HTML recibido.\\

La clase Animation cuenta con los métodos do\_onmouseover y do\_onmouseout:
\begin{itemize}
 \item do\_onmouseover(): Añade como estado solicitado el estado getting\_out del menuAutomata, indicando que al pasar el ratón 
 sobre el elemento HTML, en este caso la etiqueta 'menu' el estado solicitado sea 'saliendo' de menuAutomata..

 \item do\_onmouseout(): Añade como estado solicitado el estado getting\_in del menuAutomata, indicando que al pasar el ratón 
 sobre el elemento HTML, en este caso la etiqueta 'menu' el estado solicitado sea 'entrando' de menuAutomata.
\end{itemize}

\subsubsection{Clase MenuHandler}
\label{subsubsection:menu_handler}

Menu handler deriva de la clase device. Contiene los Gates que gestionan los eventos que se dan en cada uno de los botones 
del menú, siendo capaz de enviar los mensajes de “restart\_game”, “pause\_clock”, y “resume\_clock” que se encuentran en la 
cola de mensajes self\_events, para reiniciar el juego, pausar el reloj y reiniciar el reloj respectivamente.\\

Se comunica con la clase Galactus ya que ésta contiene las funciones attend que hacen referencia a los mensajes que es capaz 
de enviar MenuHandler. Ver código \ref{lst:code5231} (página \pageref{lst:code5231})\\

Para el despliegue del menu, creamos un objeto de la clase Animation, la cual es una subclase de Gate, para el elemento o etiqueta "menu" del HTML para 
asociar los efectos del Gate Animation al atributo menu\_effects y a partir de ello, creamos un menuAutomata.

\begin{verbatim}
 that.menu_effects = that.newGate("menu", Animation)
\end{verbatim}

Para realizar las funciones de cada botón del menú:

\begin{enumerate}
 \item Crea un Gate  para etiqueta instructions\_options del elemento HTML, sobre la cual si se recibe el 
evento do\_onclick (pulsar sobre el botón) muestra las instrucciones del juego. Ver código \ref{lst:code5221} (página \pageref{lst:code5221})

 \item Crea un Gate para la etiqueta restart\_option del elemento HTML, si se realiza un clic sobre esa opción del menú, lanza como evento el envío 
de mensaje de llamar a la función restart\_game(). Ver código \ref{lst:code5222} (página \pageref{lst:code5222})

 \item Crea un Gate para el elemento level\_option del HTML, sobre el cual si se realiza un clic,
muestra los diferentes niveles, cambiando el display de la etiqueta level\_option\_container a inline.
\end{enumerate}

La clase MenuHandler tiene dos métodos, attend\_keep\_menu\_out, attend\_keep\_menu\_in.

\begin{itemize}
 \item attend\_keep\_menu\_out(): Guarda el estado out de menuAutomata como currentState solicitado 
 (requested) para que posteriormente pase a ser current (actual) y se ejecuten las intrucciones 
 correspondientes para hacer permanecer al menú desplegado.

 \item attend\_get\_menu\_in(): Guarda el estado getting\_in de menuAutomata como currentState 
 solicitado (requested)  para que posteriormente pase a ser current (actual) y se ejecuten las 
 intrucciones correspondientes para hacer que el menú se vaya ocultando progresivamente.
\end{itemize}

\subsubsection{Clase OptionHandler}
\label{subsubsection:option_handler}

Envía mensajes al device MenuHandler para poder gestionar el despliegue del menú principal cuando se muestra el panel de niveles. 
Al igual que MenuHandler este deriva de la clase Device. A través de Levels, que se explica posteriormente, es capaz de enviar 
los siguientes mensajes:
\begin{itemize}
 \item get\_panel\_out: Despliega el panel de niveles.

 \item keep\_menu\_out: Mantiene desplegado el menú principal.

 \item get\_menu\_in". Guarda tanto el menú principal como el panel de niveles.
\end{itemize}

Será necesaria la llamada al método newGate para la creación de un objeto Levels el cual deriva de la clase Gate. 
Ver código \ref{lst:code5241} (página \pageref{lst:code5241})


\subsubsection{Clase Levels}
\label{subsubsection:levels}

Levels pertenece a la clase Gate, se encarga de gestionar el envío de mensajes desde OptionHandler a MenuHandler
(ambos Device) mediante el uso del método fireEvent() anteriormente explicado.

\begin{itemize}
 \item do\_onmouseover: Provoca que menu se mantenga desplegado. Contiene la función fireEvent, esta pasa el 
 mensaje síncrono en este caso 'keep\_menu\_out' y el evento do\_onmouseover. Ver código \ref{lst:code5251} (página \pageref{lst:code5251})

 \item do\_onmouseout: Provoca que cuando se retire el foco del submenu de niveles, este se oculte.
La funcion do\_onmouseout recoge el elemento html 'levels\_container' y cambia su display a none.
También envía mediante el método fireEvent el mensaje asíncrono 'get\_menu\_in', y el evento do\_onmouseout. 
Ver código \ref{lst:code5252} (página \pageref{lst:code5252})
\end{itemize}


\subsection{Clase Clock}
\label{subsection:clock}

El principal motivo de su creación fue la necesidad de poder cronometrar el tiempo de juego con una cuenta atrás.
Clock deriva de la clase Device y cuenta con los siguientes atributos:

\begin{itemize}
 \item start\_time: A partir de la función get\_now (función que obtiene las horas, minutos del sistema, los pasa a segundos y devuelve su suma para 
obtener el tiempo actual en segundos.) almacena la hora actual en segundos.

 \item total\_time: Almacena el tiempo en segundos del que se dispone para jugar (es recibido como parámetro).
 
 \item remaining\_time: Segundos que quedan por jugar. Queda inicializado a total\_time.
 
 \item before: Almacena en segundos el momento en el que se empezó a jugar.
 
 \item running: Variable bandera que hace hace referencia al estado el marcha del reloj. Inicializado a true.
\end{itemize}

Desde initialize llamaremos al contructor de Device para que la clase Clock sea una de sus subclases, heredando sus métodos y atributos.\\

Métodos:

\begin{itemize}
 \item reset(): Reinicia el reloj. Llama al método pause, y reinicializa el valor de remaining\_time a total\_time.

 \item get\_count(): Devuelve los segundos que quedan por jugar (remaining\_time). Es decir, devuelve el valor de remaining\_time.

 \item run(): Recalcula el tiempo que queda por jugar. Llama a la función get\_now, para conocer el momento actual. Este valor es 
 almacenado en la variable now, recalcula el valor de remaining\_time (segundos que quedan por jugar) en función del momento actual:

        tiempo que queda = al tiempo que quedaba - tiempo que ha pasado.
        
Si remaining\_time es menor o igual que cero, pone con valor falso el atributo running, es decir, el reloj deja de estar en marcha.

 \item pause(): Pausa el tiempo, cambiando el valor de running a falso.

 \item resume(): Continua la cuenta atrás desde el momento que fue pausado.Reinicializa start\_time al momento actual, a través de 
 la función get\_now. A su vez establece el valor de running a true y por último cambia el valor de total\_time por el valor de 
 remainig\_time, el cual en ese momento guarda los segundos que quedaban por jugar cuando el reloj fue pausado

 \item get\_string(): Devuelve una cadena con el tiempo restante con formato mm:ss. Si running es igual a verdadero, es decir, 
 si el reloj esta en marcha, se llama al método run y obtenemos el valor de los minutos y de los segundos utilizando las 
 funciones Floor() y Round() de la librería Math de javascript. Posteriormente se realiza la concatenación de cadenas para 
darle formato. Si por el contrario el reloj no está en marcha(running es falso), realizamos que acabamos de indicar sin 
llamar al método run().
\end{itemize}

