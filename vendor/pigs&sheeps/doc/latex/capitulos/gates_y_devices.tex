\section{Gates y Devices}
\label{sec:gates_devices}

\subsection{Introducción}
\label{subsection:intro_gates}

Device: Provee un mecanismo asíncrono de comunicación. Usa una cola de mensajes y dispara eventos. No tiene conexión propia con el DOM de HTML,
pero ésta se realiza mediante un objeto de la clase Gate.\\

Gate: Es un “envoltorio” para elementos html.\\

Están diseñados para responder a eventos html manteniendo el campo de aplicación objeto. Recibe como parámetros opcionales el elemento html a 
envolver (por ejemplo un div), el contenedor html donde situar el gate y las acciones de respuesta.\\

Por ejemplo, si en el documento html tenemos el siguiente elemento:
\begin{verbatim}
     <div id="button_bar">&nbsp;</div>
\end{verbatim}
Escribiríamos:
\begin{verbatim}
     var brush_button = new Gate("brush_btn", "button_bar")
\end{verbatim}
Lo que generaría un div con id:”brush\_btn” y un objeto javascript que responderá a cada evento html que se haya definido en el correspondiente 
handler. Ver figura 5.1.1

\begin{figure}[p]
Figura 5.1.1
\begin{verbatim}
     Button.prototype = new Gate
     Button.prototype.constructor = Button

     function Button(element){

        try {
            if (arguments.length)
                Gate.call(this, element)// LLama al superconstructor de Gate.
        } catch (e) {
            if (\$K_debug_level >= \$KC_dl.DEVELOPER)
                alert("No event handlers were found.\nException: " 
                       + e.toSource())
        }
     }

     Button.prototype.do_onclick   = function(event, element){
        alert("You have made click.")
     }

     var b = new Button(id_of_html_element, null, {
         do_onmouseover: function(event, element) {
             alert("Hello")
         }
     })
\end{verbatim}
\end{figure}

\subsection{Menú y Submenú}
\label{subsection:menu}

MenuAutomata: Controla los estados de entrada y salida del menú. Deriva de la clase threadAutomata y tiene una serie de estados posibles:\\
\"out", "getting\_out", "getting\_in", \"inside".\\
Estos estados, a su vez, pueden encontrarse en estado anterior (previous), actual (current) o solicitado (requested). Según cual sea  el estado 
actual (current), se realizarán la funciones correspondientes.\\

Estado out: Cambia la altura de menu a 250px. Ver figura 5.2.1
 
\begin{figure}[p]
figura 5.2.1
\begin{verbatim}
[function(){
    this.gate.panel.style.height = "" + 250 +"px" 
}; ],
\end{verbatim}
\end{figure}

Estado getting\_out : Incrementa la altura de menu de 5 en 5px  mientras este mida menos de 206 px y mas de 50px. Ver figura figura 5.2.2

\begin{figure}[p]
figura 5.2.1
\begin{verbatim}
[function (){
    if( this.menu_height >= 50 && this.menu_height<=205)
       this.menu_height+=5
    this.gate.panel.style.height = "" + this.menu_height +"px"
};],
 
\end{verbatim}
\end{figure}
            
Estado  getting\_in : Decrementa la altura de menú de 5 en 5px  mientras éste mida  más de 55px. Ver figura figura 5.2.3

\begin{figure}[p]
figura 5.2.1
\begin{verbatim}
[function(){
    if( this.menu_height >=55 ){         
        this.menu_height-=5
    this.gate.panel.style.height = "" + this.menu_height +"px"
},],
\end{verbatim}
\end{figure}
        
Animation: Recoge los eventos que se producen en el menú a partir de la interacción del usuario. Animation.js deriva de la clase gate.
Recibe el elemento html que envuelve  el gateAnitamion. En su función initialize además de crear como atributo  el elemento (html) y de llamar
al super constructor de gate, asocia los efectos de menuAutomata al elemento html recibido.\\

La clase animation cuenta con los métodos do\_onmouseover y do\_onmouseout:
\begin{itemize}
 \item do\_onmouseover: Añade como estado solicitado el estado getting\_out del menuAutomata.
Es decir, el estado "saliendo" de menuAutomata.

 \item do\_onmouseout: Añade como estado solicitado el estado getting\_in del menuAutomata.
Es decir, el estado "entrando" de menuAutomata.
\end{itemize}

MenuHandler: Se encarga de disparar los eventos que se producen en el menú. Se comunica con la clase OptionHandler.
Menu handler deriva de la clase device, éste  tiene sus propios eventos : self\_events.\\

Para el despliegue del menu, creamos un gate contenido en Animation para el elemento o etiqueta "menu" del html para asociar los efectos del 
gate Animation al atributo menu\_effects y apartir de ello creamos un menuAutomata.

Para realizar las funciones de cada botón del menu:
\begin{itemize}
 \item Crea un gate  para etiqueta instructions\_options del elemento html, sobre la cual si se recibe el 
evento do\_onclick (pulsar sobre el botón) muestra las instrucciones del juego

 \item Crea un gate para la etiqueta restart\_option del elemento html, si se realiza un click sobre esa opción del menu, lanza como evento el envio 
de mensaje de llamar a la función restart\_game()

 \item Crea un gate para el elemento level\_option del html, sobre el cual si se realiza un click,
muestra los diferentes niveles, cambiando el display de la etiqueta level\_option\_container a inline.
\end{itemize}

La clase menuHandeler tiene dos métodos, attend\_keep\_menu\_out, attend\_keep\_menu\_in.
\begin{itemize}
 \item attend\_keep\_menu\_out: Guarda el estado out de menuAutomata como estado actual solicitado.

 \item attend\_get\_menu\_in: Guarda el estado getting\_in de menuAutomata como estado actual solicitado

 \item OptionHandler: Dispara los eventos que se producen en la clase Levels y se comunica mediante mensajes con la clase MenuHandler.
\end{itemize}

Al igual que menuHandler este Deriva de la clase Device  y tiene diferentes posibles estados:
“get\_panel\_out", "keep\_menu\_out", "get\_menu\_in".
Éste crea diferentes Gate para cada botón level.

- Levels: Recoge los eventos que se generan en el submenú. 
Deriva de la clase gate y tiene como métodos do\_onmouseover y do\_onmouseout:

\begin{itemize}
 \item do\_onmouseover: Lanza como evento la creación de un nuevo mensaje de mantener el menu desplegado

 \item do\_onmouseout: Oculta el elemento level\_option\_container, cambiando su display a none y lanza como evento un nuevo mensaje con el estado 
get\_menu\_in para ocultar el menu.
\end{itemize}


\subsection{Reloj}
\label{subsection:reloj}

Reloj: Deriva de la clase device y cuenta con los siguentes atributos:

\begin{itemize}
 \item start\_time: Apartir de la función get\_now (función que obtiene las horas, minutos del sistema, los pasa a segundos y devuelve su suma para 
obtener el tiempo actual en segundos.) almacena la hora actual en segundos.
 \item total\_time: Almacena el tiempo en segundos del que se dispone para jugar (es recibido como parámetro)
 \item remaining\_time: Segundo que quedan por jugar. Queda inicializado a total\_time.
 \item before: Almacena el en segundos el momento en el que se empezo a jugar.
 \item running: Hace referencia al estado el marcha del reloj. Inicializado a true.
\end{itemize}

La función Initialize llama constructor de device.

Métodos de la clase Clock:

\begin{itemize}
 \item Reset: Reinicia el reloj. LLama al método pause, y reinicializa el valor de remaining\_time a total\_time.

 \item get\_count: Devuelve los segundos que quedan por jugar (remaining\_time).

 \item run: Recalcula el tiempo que queda por jugar. Vuelve a llamar a la función get\_now, para conocer el momento actual, este valor es almacenado 
en la variable now, recalcula el valor de remaining\_time (segundos que quedan por jugar) en función del momento actual:

        tiempo que queda = al tiempo que quedaba - tiempo que ha pasado.
        
Si remaining\_time es menor o igual que cero, pone con valor falso el atributo running, es decir, el reloj deja de estar en marcha.

 \item pause: Pausa el tiempo, cambiando el valor de running a falso.

 \item resume: Continua la cuenta atrás desde el momento que fue pausado. Reinicializa start\_time apartir del momento actual, atraves de la 
 función get\_now. Recalcula remaining\_time.

        lo que queda = lo que quedaba - lo que ha pasado.

Establece el valor de running a true.

 \item get\_string: Devuelve una cadena con el tiempo restante con formato mm:ss.
\end{itemize}

Si ranning es igual a verdadero, es decir, si el reloj esta en marcha, se llama al método run y obtenemos el valor de los minutos y de los 
segundos utilizando las funciones floor y round de la librería math de javascript. Posteriormente se realiza la concatenación de cadenas para 
darle formato. Si por el contrario el reloj no esta en marcha(running es falso), realizamos las mismas operaciones sin llamar al método run.

