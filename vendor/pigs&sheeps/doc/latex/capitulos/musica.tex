\section{Música}
\label{sec:musica}

\subsection{Introducción}
\label{subsection:intro_musica}

Desde la función playSound de Galactus se crea un objeto AudioContext  el cual controla la ejecución del procesamiento del audio.
En esta misma función se crea un objeto de la clase bufferLoader. 

En su creación pasamos como parámetros: 
\begin{itemize}
 \item El objeto AudioContext
 \item Un array que contiene las rutas de los archivos de audio que se quieren reproducir.
 \item La llamada a la función  finishedLoading.
\end{itemize}

La función finishedLoading() se encargará de crear y reproducir los recursos, recogiéndolos del array que contiene las rutas. 
Esta función será ejecutada después de solicitar la ejecución del método load de bufferLoader.

\subsection{Clase BufferLoader}
\label{subsection:buffer_loader}

Su principal función es la de cargar las pistas de música en un buffer de datos.
Para que ello se lleve a cabo, hay que ejecutar su método load(). Este método se encarga de llamar al método loadBuffer para cada elemento del array con las URLs de las pistas de 
canciones que se envían como parámetro, mandando éstas así como su posición en el array. \\

Este último método cargará las pistas de audio en el buffer, realizando una petición HTTP al servidor y descodificará el audio como respuesta, añadiendo esa pista recibida al 
buffer y, posteriormente, ejecutará la llamada a la función finishedLoading (la cual indicamos que tiene que ser llamada en la creación de la clase bufferLoader()), en el caso de 
no haberse encontrado ningún error.\\

Por último la función stopPlaying de Galactus el método stop para la pista de música.
