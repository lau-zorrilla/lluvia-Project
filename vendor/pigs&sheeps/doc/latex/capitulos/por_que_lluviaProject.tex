\section{Por qué Lluvia Project}
\label{section:por_que}

\subsection{¿Qué es lluviaProject?}
\label{subsection:que_es}

Es una API Open Source de javascript  que incorpora parte de las funciones nativas de Ruby. Soporta multihilos a pequeña escala y provee un 
sistema de mensajes. Las aplicaciones deben crearse dentro del directorio vendor. Dentro del archivo html es necesario incluir en la 
etiqueta body llamar a la función Bring LLuvia().\\

La aplicación debe de contar con un archivo generalmente llamado dependencies.js, donde deben incluirse el nombre de los ficheros de los que 
depende la aplicación. Si existe una función main esta se llama automáticamente después de cargarse lluvia.\\

Ver ejemplo en figura 2.1.1
\begin{figure}[p]
\begin{verbatim}
Figura 2.1.1

$K\_app\_dependencies = [
    {   module: "Boids", 
        description: "Boids Demo App.",
        path: "",
        files: [
            { name: "brain/behavior\_modifier.js",  
               description: "Self protection behaviors." },
            { name: "brain/behavior.js",           
               description: "Abstract Behavior." },
            { name: "brain/security\_behavior.js",  
               description: "Self protection behaviors." },
            { name: "brain/itinerant\_behavior.js", 
               description: "Definition of itinerant behaviors." },
            { name: "brain/behavior\_group.js",     
                description: "Group of related behaviors." },
            { name: "brain/brain.js",              
               description: "Boid Brain." },
            { name: "boid.js",                     
               description: "One Boid." },
            { name: "world\_interface.js",          
               description: "World Interface." },
            { name: "boid\_editor.js",              
               description: "Boid panel editor." },
            { name: "world.js",                    
               description: "The world where all boids live." },
            { name: "main.js",                     
               description: "main function." },
        ]
    }
]
\end{verbatim}
\end{figure}



\subsection{Ventajas}
\label{subsection:ventajas}


Una de las funcionalidades que ofrece lluviaProject es el objeto boid. A este objeto se le puede configurar tanto el comportamiento como las 
características físicas de aquello que se quiere representar.\\

En la parte física, se puede modelar tanto el peso, la visión, la velocidad, la posición, la aceleración, capacidad de frenada, la capacidad 
de giro, etc. Y por otra parte, se pueden modelar comportamientos de huida, de persecución, de cohesión, de alineamiento, de separación e 
incluso comportamientos hechos a medida. Esta característica de lluviaProject permite representar los comportamientos de animales, objetos 
o partículas, entre otros.\\

Esto, junto con el hecho de que añade librerías y funciones que hacen que la escritura de código en javascript sea más cómodo, nos hizo 
decidirnos por lluviaProject.\\
