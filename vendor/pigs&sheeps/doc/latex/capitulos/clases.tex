\section{Clases}
\label{sec:clases}

\subsection{Estructura de una clase}
\label{subsection:estructura}

\lluvia, al igual que otros lenguajes orientados a objetos, está organizado en clases.\\

Para crear una clase es necesario definir una función constructora. 
\begin{verbatim}
 var myClass = function() {;} 
\end{verbatim}

Ésta puede contener atributos y métodos públicos accesible sólo desde la clase, que no se pueden heredar y que se definen como:
\begin{verbatim}
myClass.attribute = VALUE
myClass.method = function(){;}
\end{verbatim}

También se pueden definir atributos y métodos heredables por aquellos objetos que derivan de la clase. 
Para esto, es necesario añadir prototype entre el nombre de la clase y el nombre del atributo o método. 
\begin{verbatim}
myClass.prototype.attribute = VALUE
myClass.prototype.method = function(){;}
\end{verbatim}

Las variables y métodos privados se definen dentro de la función como
\begin{verbatim}
var attribute = VALUE
\end{verbatim}

Una clase puede devolver una variable o un objeto complejo, llamado cierre, con capacidad para acceder a más de un atributo. 
De esta manera, una clase puede ser, a su vez, una factoría de objetos.
\begin{verbatim}
var derived_class = new myClass()
\end{verbatim}

Estas clases pueden aceptar parámetros o no y new es opcional.\\

Aunque \lluvia{} no define nada al respecto, los cierres se suelen usar a través de la función yield, la cual busca entre los parámetros 
un objeto función y lo ejecuta.\\

Las clases aceptan un objeto de configuración, el cual permite modificar la propia clase en caso necesario. Al igual que en javascript, 
no es necesario definir el tipo de las variables. Ver ejemplo en código \ref{lst:code411} (página \pageref{lst:code411})\\

\lluvia{} permite definir una función dentro de la propia clase. Ver ejemplo en código \ref{lst:code412} (página \pageref{lst:code412})\\

La palabra reservada this, dentro del ámbito de new, hace referencia al objeto que se está creando, mientras que fuera de él hace 
referencia al objeto que se está ejecutando.\\

\textbf{Object\#method\_missing:} \\
Cuando una función no existente es llamada, se levanta una excepción y se llama al método method\_missing()



\subsubsection{Cadena Prototípica}
\label{subsubsection:prototipos}

Las clases siguen las convenciones de las cadenas prototípicas de Javascript. Esto quiere decir que si el atributo no está presente en el 
objeto, entonces se mira en el constructor de la clase padre. Si ésta deriva a su vez de otra, entonces se busca en el constructor del 
padre de la segunda. Ver figura figura \ref{fig:figura413} (página \pageref{figura:figura413})\\

Aquellas funciones definidas sin la palabra reservada prototype no son accesible por ninguna instancia de la clase. Tanto éstas como aquellas 
funciones que no que no acceden a los atributos de clase se consideran estáticas.\\

Es posible definir métodos singleton, que se crean en el constructor de la clase y que no pertenecen a la clase padre. Dentro de estos 
métodos, this pierde su significado y es necesario que esté recogido en una variable local para poder acceder a él.\\



\subsection{Clase Boid}
\label{subsection:boid}

\subsubsection{Introducción}
\label{subsubsection:boid_general}

Un Boid se define como un carácter personal autónomo que recibe un objeto de configuración y un bloque. Al crear una nueva instancia, se 
le puede pasar como parámetro un objeto de configuración y/o un bloque de código. Si se le pasa un objeto de configuración, es necesario 
definir todos los parámetros a incluir en el constructor del nuevo Boid. Si se le pasa un bloque, se pueden definir sólo aquellos parámetros 
que vayan a cambiar con respecto a la clase padre.\\

La configuración por defecto incluye un objeto geo\_data, el cual contiene los atributos de posición, velocidad y aceleración, color del 
boid, un cerebro, velocidad máxima, masa, el objeto vision, con radio y ángulo, y el objeto límites dinámicos, que incluye capacidad de 
empuje, de giro y de frenada. Ejemplo de configuración por defecto. Código \ref{lst:code4211} (página \pageref{lst:code4211})\\

Es posible modificar su posición, velocidad o aceleración en tiempo real según sea necesario. También se le puede asignar un comportamiento 
de los que ya existen o crear nuevos que cubran necesidades más específicas.\\

Cada Boid tiene acceso directo a la descripción geométrica de la escena, pero los comportamientos de manada requieren que éste reaccione 
sólo a aquellos boids que se encuentren dentro de un área específica alrededor de él. Este área se define por una distancia, medida desde 
el centro del Boid, y un ángulo, medido en la dirección de avance del Boid. Aquellos Boids que se encuentran fuera de este área son ignorados. 
Ver figura \ref{fig:figura4212} (página \pageref{fig:figura4212})\\


Métodos:
\begin{itemize}
 \item position():
Obtiene o define la posición del Boid

  \item velocity():
Obtiene o define la velocidad del Boid

  \item acceleration():
Obtiene o define la aceleración del Boid

  \item start():
Guarda la hora que marca el procesador.

  \item delta\_at():
Tiempo que ha pasado desde la última vez que la variable 'last time' fue actualizada

  \item update\_physics():
Calcula la nueva posición, velocidad y aceleración en función del tiempo que ha pasado.

  \item run():
Actualiza el tiempo en las variables del Boid.

  \item first\_draw():
Pinta el Boid y lo guarda como imagen, lo que ahorra tiempo de cálculo.

  \item draw():
Pinta un Boid en un mundo definido por un contexto.

  \item heading():
Alinea el vector normal con el último vector de dirección del Boid.

  \item locale():
Expresa las coordenadas locales del sistema en coordenadas globales.

  \item globale():
Expresa las coordenadas globales del sistema en coordenadas locales.

  \item localize():
Cambia las coordenadas globales del sistema en coordenadas del Boid.

  \item globalize():
Cambia las coordenadas locales del sistema en coordenadas del mundo.

  \item visible\_objects():  
Pregunta al mundo si existe un objeto visible dentro de las habilidades de visión y las coordenadas de velocidad, aceleración y posición.
del Boid.

  \item requested\_acceleration(): Devuelve al mundo la aceleración requerida (lo que el cerebro desea y lo que el cuerpo permite).

  \item clip():
Ajusta la aceleración en función de los límites dinámicos del Boid.

  \item set\_target():
Define un target.

  \item  target\_data():
Devuelve la información del target.
\end{itemize}



\subsubsection{Clase Brain}
\label{subsubsection:brain}

Es la clase que crea un cerebro para el Boid. En ella, se guardan los comportamientos disponibles para ser activados. Estos 
comportamientos conocidos se almacenan en la variable \textit{catalog}. El cerebro, además, se encarga de discriminar entre aquellos que 
pueden ser activados por un Boid concreto y los que no.\\

Guarda las aceleraciones que posee el Boid para cada uno de sus comportamientos activos. De esta manera, se puede calcular la aceleración que 
el Boid necesita para cambiar su posición en función del objetivo.\\


Métodos:
\begin{itemize}
\item can\$U():
Comprueba si el Boid puede activar un comportamiento específico.

\item can\_be\_in\$U():
Comprueba si un comportamiento puede ser activado para un Boid.

\item activate():
Activa un comportamiento determinado en un Boid.

\item deactivate():
Desactiva un comportamiento determinado en un Boid.

\item is\_in\$U():
Comprueba si un comportamiento está en la lista de aquellos aceptados por el Boid.

\item get\_behavior():
Obtiene el comportamiento pasado como parámetro

\item \$see\_accelerations():
Muestra las aceleraciones guardadas por el método desired\_accelerations()

\item desired\_accelerations():
Guarda las aceleraciones de un determinado Boid para todos sus objetivos

\item desired\_acceleration():
Calcula la aceleración que el Boid necesita para cambiar su posición.
\end{itemize}



\subsubsection{Clase Sheep}
\label{subsubsection:sheep}

La clase oveja deriva de la clase Boid, por lo que hereda la mayoría de su funcionalidad. La diferencia fundamental con la clase padre es 
la creación de nuevas funciones y la redefinición de métodos ya existentes.\\

Métodos:
\begin{itemize}
\item sheep\_limits(): Define los límites entre los que los Boids oveja se pueden mover. Esta función es necesaria para que éstos no se 
salgan de los límites impuestos por el canvas. Ver código \ref{lst:code4231} (página \pageref{lst:code4231})

\item  update\_physics(): Calcula la nueva posición, velocidad y aceleración en función del tiempo que ha pasado. Una de las razones para 
redefinirla es poder llamar a la función sheep\_limits(). Esta función, además, comprueba si el Boid oveja ha entrado en el corral y, en 
caso afirmativo, suma un punto al contador. Ver código \ref{lst:code4232} (página \pageref{lst:code4232})

\item  first\_draw():  Pinta el Boid y lo guarda como imagen, lo que ahorra tiempo de cálculo. Se redefine para evitar que los Boids se 
pinten como círculos, ya que en el constructor se determina una imagen para representar a las ovejas. Ver código \ref{lst:code4233} (página \pageref{lst:code4233})

\item  draw(): Pinta un Boid en un mundo definido por un contexto. La diferencia con el método original es que ya no se pintan líneas en 
el canvas. Si la velocidad es mayor que cero, se carga la imagen de la oveja mirando hacia la izquierda y si no, hacia la derecha. 
Ver código \ref{lst:code4234} (página \pageref{lst:code4234}).

\end{itemize}



\subsubsection{Clase Pig}
\label{subsubsection:pig}

La característica principal de esta clase es su capacidad para responder a las coordenadas del ratón cuando el usuario dispara el evento 
onClick(). Éstas son capturadas por el objeto MouseCoordinates, el cual deriva de la clase Gate. A través de su método do\_onclick(), 
transforma las coordenadas globales de la pantalla en locales del canvas y las guarda en sendas variables de clase. Estas variables se 
pasan a la clase World como parámetros de su función move\_shepherd() y se usan para definir el objetivo del Boid, el cual tiene programado un 
comportamiento de persecución. Ver figura \ref{fig:lock} (página \pageref{fig:lock})\\ 

Métodos:\\

Esta clase cuenta con dos métodos, modificados de la clase original. Al igual que en la clase Sheep, ha sido necesario redefinir 
first\_draw() y draw() para poder cargar la imagen del cerdito en lugar de pintar el círculo que se crea por defecto para representar el 
Boid.



\subsection{Clase Behavior}
\label{subsection:behavior_section}

\subsubsection{Introducción}
\label{subsubsection:behavior}

La clase Behavior permite modelar el comportamiento de los Boid. Esto se consigue modificando el vector de aceleración que cada uno de los 
Boids posee como parte de su estructura interna. Un comportamiento se puede definir como un objeto que devuelve una aceleración deseada en 
un momento concreto.\\

Al ser una clase abstracta, no permite crear instancias, de manera que para crear un comportamiento nuevo es necesario generar una clase 
nueva que derive de ésta.

\begin{verbatim}
Modo incorrecto de crear un comportamiento:

var a = new Behavior()

Modo correcto:

NewBehavior.prototype = new Behavior\\
\end{verbatim}

Estos comportamientos puede tener pre y post modificadores. Los primeros modifican la aceleración antes de devolverla y los segundos,
después. Un ejemplo de premodificador es forsee, que se encarga de calcular dónde estará el objetivo en la próxima medida de tiempo. Este 
premodificador unido al comportamiento de búsqueda seek hacen que varíe la aceleración para modificar la trayectoria y perseguir al objetivo. 
Un ejemplo de premodificador sería arrival, el cual va disminuyendo la aceleración a medida que se acerca al objetivo.\\

Los comportamientos disponibles para los Boids en este momento son seek (búsqueda), flee (huida), wander (paseo), containment (contención 
en el espacio), alignment (alineamiento con otro Boids), cohesion (acercamiento a otros Boids), separation (separación de otros Boids) y 
obstacle avoidance (sorteo de obstáculos).\\

Métodos:
\begin{itemize}
\item decompose\_name():
Dada una cadena de caracteres que representa el nombre del comportamiento y sus modificadores, devuelve un array con la lista de pre
modificadores, el nombre del comportamiento y una lista con los post modificadores.

\item catalog():
Lista todos los comportamientos y modificadores conocidos.

\item new():
Crea un nuevo comportamiento.

\item type\_of():
Devuelve la clase asociada con el nombre del comportamiento.

\item is\_a\$U():
Comprueba si el nombre pasado como parámetro se corresponde con uno de los comportamientos existentes.

\item desired\_acceleration():
Devuelve un vector con la aceleración deseada de ese comportamiento.

\item is\_premodified\_by\$U():
Devuelve true si el nombre del modificador está en la lista de pre modificadores

\item is\_postmodified\_by\$U():
Devuelve true si el nombre del modificador está en la lista de post modificadores.

\item is\_modified\_by\$U():
Devuelve true si el nombre del modificador está tanto en la lista de pre como de post modificadores.

\item get\_modifiers\_for():
Devuelve una lista de pre o post modificadores.

\item all\_modifiers():
Devuelve una lista de todos los modificadores.

\item activate\_modifier():
Activa un modificador determinado.

\item get\_modifier():
Busca un modificador determinado y devuelve null si no lo encuentra.

\item deactivate\_modifier():
Desactiva un modificador determinado.
\end{itemize}


\subsubsection{Clase Sheep Behavior}
\label{subsubsection:sheep_behavior}

Es la clase encargada de generar el comportamiento de las ovejas. Deriva de la clase Behavior. Cuando el cerdito oveja entra dentro del 
radio de visión de la oveja, esto despierta en ella un comportamiento de huida. Es decir, aumenta su aceleración hasta su límite para 
disminuir ésta gradualmente a medida que se va alejando del Boid cerdito.\\

Éste es el primero de los cinco comportamientos que se quieren implementar (véase cuadro \ref{table:nonlin} en la página \pageref{table:nonlin}). Los otros cuatro quedan para una fase 
posterior de desarrollo.\\

Métodos:
\begin{itemize}
\item set\_target():
Define el objetivo del Boid, que debe ser otro Boid y el cual se pasa como parámetro.

\item target\_data():
Define la información sobre la posición del Boid objetivo.

\item get\_target():
Devuelve la posición del objetivo.

\item target\_at():
Da la distancia entre el Boid y su objetivo.

\item desired\_velocity():
Devuelve la velocidad deseada del Boid. Ver código \ref{lst:code4322} (página \pageref{lst:code4322})

\item desired\_acceleration():
Devuelve la aceleración deseada del Boid.
\end{itemize}


\subsubsection{Clase Alignment Behavior}
\label{subsubsection:alignment_behavior}

Al comportamiento básico de la oveja se le añadió el comportamiento de alineación, con el objetivo de crear una sensación de grupo más 
marcada. El comportamiento de Alignment está dentro de los comportamientos de manada del cerebro.\\

Se calcula el alineamiento medio como la media de las velocidades. Es necesario que aumente la aceleración para eliminar la componente 
normal al alineamiento en la velocidad del Boid, es decir, suprime toda la velocidad que no esté en la dirección del alineamiento. 
Ver código \ref{lst:code4331} (página \pageref{lst:code4331})\\ %Falta!!!!!!!!!!!!!!!!!!!!!!!!!!!!!!!



\subsubsection{Clase Seek Mouse Behavior}
\label{subsubsection:seek_mouse_behavior}

Deriva de la clase Seek Behavior y modifica el comportamiento de la clase Pig a través de uno de sus métodos, al que se le pasan 
como parámetros las variables x e y, que se corresponden  con las coordenadas x e y del ratón, capturadas y transformadas en coordenadas 
locales por la clase MouseCoordinates.\\

Este comportamiento se activa en la clase Galactus, en la creación del Boid cerdito. Después, en la función move\_shepherd() de la clase 
World, se consigue el comportamiento de seek mouse, se obtienen los valores de x e y y se escalan en función de la perspectiva del canvas.
Estas coordenadas escaladas se pasan como parámetro a la clase Seek Mouse a través de su método  set\_target\_at(). 
Ver código \ref{lst:code4341} (página \pageref{lst:code4341})\\



\subsection{Clase World}
\label{subsection:world}

La clase World deriva de la clase Device y es la que genera el mundo en el que viven los Boids. Necesita un objeto canvas para poder 
existir, aunque, en principio, sus límites son infinitos. Esto quiere decir que el usuario verá en pantalla tan sólo una parte del mundo, 
mientras que los Boids pueden representarse en cualquier parte de él, saliendo del canvas en muchas ocasiones.\\

Las funciones principales del mundo son gestionar el canvas y los boids existentes. A través de diferentes funciones, se puede modificar 
el tamaño del mundo o del canvas, guardar los Boids creados en una lista y acceder a ésta o actualizar los datos que reciben estos Boids 
para modificar su estado.\\

Además, el mundo puede para enviar y recibir mensajes, una de las características de la clase Device. En la variable de clase 
this.self\_events, de tipo Array, están guardados los mensajes que el mundo puede enviar. Puede recibir cualquier mensaje siempre y cuando
haya definida una función attend\_nombre\_del\_mensaje().\\

Métodos:
\begin{itemize}
 \item set\_dashboard():
Define una nueva interfaz del mundo en el canvas.

 \item width():
Define el ancho del mundo.

 \item height():
Define el alto del mundo.

 \item screen\_width():
Define el ancho del canvas.

 \item assert\_screen():
Comprueba si se ha creado un canvas.

 \item screen\_height():
Define el alto del canvas.

 \item has\_born():
Registra la creación de un Boid y se envía un mensaje de notificación a sí mismo.

 \item get\_boids():
Crea un array con todos los Boids.

 \item each\_boid():
Permite acceder a cada boid guardado en el array de Boids

 \item start():
Inicializa el tiempo de los boids

 \item is\_initalized():
Comprueba si el mundo está inicializado. Si es así, el usuario está reiniciando el juego.

 \item draw():
Dibuja el canvas en el mundo.

 \item move\_shepherd():
Define el target del Boid en función de las coordenadas del ratón. Ver código \ref{lst:code441} (página \pageref{lst:code441})

 \item step():
Actualiza la fecha en los cálculos de la física de los Boids.

 \item is\_one\_second\_from\_begining():
Define la hora de comienzo del mundo en un segundo más de la hora en la que la función fue llamada.

 \item show\_boids():
Muestra los Boids que existen actualmente en el mundo.

 \item check\_level():
Comprueba si el nivel ha sido superado por el usuario. Si es así, para el mundo y el cronómetro y llama a la función que cambia la imagen 
de fondo del canvas. Ver código \ref{lst:code442} (página \pageref{lst:code442})

 \item running\_steady():
Actualiza la hora del procesador, comprueba el nivel y si éste ha sido superado, deja de pintar el canvas. Ver código \ref{lst:code443} (página \pageref{lst:code443})

 \item visible\_for():
Crea un array con los Boids que el Boid referenciado puede ver.

 \item new\_boid():
Crea un nuevo Boid.

 \item new\_seeker():
Crea un nuevo Boid con el comportamiento de búsqueda prefijado.

 \item start\_and\_run():
Inicia en mundo y lo pone en marcha.

 \item attend\_focus\_boid():
Actualiza la cola de mensajes.

 \item new\_boid\_of():
Crea un nuevo Boid de una subclase de Boid.

 \item method\_missing():
Provee el método dinámico new\_boid\_as\_(ClassName)
\end{itemize}


\subsection{Clase Galactus}
\label{subsection:galactus}

Clase que se encarga de gestionar la creación de un nuevo mundo al principio de cada juego. Además, se encarga de destruir el mundo 
anterior cada vez que se reinicia la partida. La función principal de está clase es start\_world(), donde se crea el mundo nuevo, se añade 
un puerto de escucha de mensajes y se llama a las funciones que controlan la cuenta atrás y la música. Además, se crea el Boid cerdito y 
los Boid oveja y se les pasan los parámetros de configuración y se definen los comportamientos específicos.\\

La función playSound() se encarga de gestionar el sonido del juego y todo lo relacionado con él. Countdown() es responsable de reloj de 
cuenta atrás, de iniciarlo cuando se inicia el mundo, pararlo, etc. Destroy\_world() y attend\_restart\_world() son las funciones encargadas 
de gestionar la destrucción del mundo y creación de uno nuevo cuando el usuario pulsa el botón de reiniciar el juego. 
Attend\_restart\_world() es la función que responde al mensaje 'restart\_world' enviado por el mundo gracias al puerto añadido en la 
función start\_world().\\

Métodos:
\begin{itemize}
 \item start\_world():
Inicia el mundo. Ver código \ref{lst:code451} (página \pageref{lst:code451})

 \item playSound():
Inicia el sonido del juego respondiendo al evento onClick() del botón de play

 \item countdown():
Inicia la cuenta atrás del temporizador de la pantalla. Si el tiempo se termina, se llama a la función que pinta en el canvas la imagen de final del juego

 \item destroy\_world():
Destruye el mundo actual y deja saber al nuevo mundo que ya había un mundo creado antes.

 \item attend\_restart\_game():
Reinicia el mundo. Atiende al mensaje enviado por el gate cuando alguien pulsa el botón restart del juego.
\end{itemize}
